\documentclass[a4j,12pt]{jarticle}

% \usepackage{showkeys}
\usepackage[dvips]{color,graphicx}
\usepackage{amsmath}
\usepackage{amssymb}
%\usepackage[utf8]{inputenc}
%\usepackage[T1]{fontenc}
%\usepackage{times}
%
\usepackage{amsmath,amssymb}
\usepackage{bm}
\usepackage{graphicx}
\usepackage{verbatim}
%
%\makeindex
%
%\setlength{\textwidth}{\fullwidth}
%\setlength{\textheight}{40\baselineskip}
\addtolength{\textheight}{\topskip}
\setlength{\voffset}{-0.55in}
%
\setcounter{tocdepth}{4}
\renewcommand{\contentsname}{\large \centerline{Contents}}

%
\title{@Home SPLルールブック}
\author{Family \& Robotics Association}
\date{\today}

\begin{document}
%
%
\maketitle

ルールに関する問い合わせ先: {\tt tc@familyrobotics.org}
\vspace{3em}
%
%
\tableofcontents
\newpage

\section{はじめに}

\subsection{このルールブックについて}

このルールブックは、RoboCup @Homeに標準ロボットリーグを
設ける提案及び、プレ競技会開催のために書かれるものである。
このルールブックの執筆・管理は、
Family \& Robotics AssociationのTechnical Comittee
(TC)によって行われる。

\section{ロボットの規格}

本リーグは、各チームがほぼ同一の安価なハードウェアを用いる事によって、
ハードウェア調達のコスト・組み立てやチューニングの手間を減らし、
ソフトウェア開発に集中できるようにするために開催するものである。

ロボットは、台車、ロボットハンド、Kinect for Xbox 360、及び計算機で構成されることとする。
それらを動作させるための配線や基盤以外のものをロボットに搭載してはならない。
台車、ロボットハンドについては以下のものに限定する。
\begin{itemize}
	\item 台車: Turtlebot2\footnote{http://www.rt-shop.jp/index.php?main\_page=product\_info\&cPath=23\&products\_id=758}
	\item ロボットアーム: Turtlebot2用ロボットアーム ``CRANE'' (クライン)\footnote{http://www.rt-shop.jp/index.php?main\_page=product\_info\&cPath=1\&products\_id=1307}
\end{itemize}
これら商品は、取り扱い説明書の通りに正確に組み立てられていなければならない。

計算機については、Turtlebot2の天板に搭載するものとする。
個数の制限は設けない。寸法については、天板からのはみ出しを50[mm]
以内


todo: なにか制限は?

todo: 寸法等の資料


\section{その他}

\subsection{リーダーミーティング}

本リーグは、各チームが主体的に運営することとし、
競技会中は、全チームのリーダーが集まったリーダーミーティングが
最も重要な組織である。
リーダーミーティングの議決は本書のルールの
一時的な改変等、ほぼ全ての事項について効力を発揮する。
ただし、効力の発揮にはTCの了承が必要である。
また、議決の方法も
リーダーミーティングにおいて決定されなければならない。

また、ルールに関するトラブルを避けるため、
競技前にリーダーミーティングにおいてルールの
読み合わせが行われなければならない。


リーダーミーティングには、リーダーでない一般メンバー
もオブザーバーとして参加できる。
また、出席の求めがあった場合、
指名された一般メンバーは参加しなくてはならない。


競技中のトラブルの発生、あるチームの不正疑惑等が発生した場合、
TCメンバー、あるいは任意のチームのリーダーは、
リーダーミーティングを開催する事ができる。
開催のタイミングについては、緊急を要するもの以外は、
観客が入っている時間帯を極力さけるものとする。


%
\end{document}
